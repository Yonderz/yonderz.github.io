---
title: Fundamental 1-unit in Z√d
mathjax: true
tags: number-theory
excerpt:
---
\documentclass{article}
%preamble
\usepackage{parskip}
\usepackage{amsmath,amsthm,amssymb}
\usepackage[colorlinks=true]{hyperref}
\newtheorem*{thm}{Proposition}
%\newtheorem{lem}{Lemma}
%\newtheorem{claim}[lem]{Claim}
\theoremstyle{definition}\newtheorem{definition}{Definition}

\begin{document}
\begin{thm}
	Suppose there exists a nontrivial element of $\mathbb { Z } [ \sqrt { d } ] ^ { \times , 1 }$.
	Then every element of $\mathbb { Z } [ \sqrt { d } ] ^ { \times , 1 }$ is of the form $\pm \epsilon ^ { n }$ for some $n$ in $\mathbb { Z }$, where $\epsilon$ is the fundamental 1-unit.
\end{thm}	

\begin{proof}
	Assume there exists a nontrivial element of $\mathbb { Z } [ \sqrt { d } ] ^ { \times , 1 }$, $t$. $\forall n \in \mathbb {Z}, t \neq \pm \epsilon ^ { n } $. Then $\exists k \in \mathbb { Z } : t \in (\epsilon^k,\epsilon^{k+1}) \, \lor \, -t \in (\epsilon^k,\epsilon^{k+1})$.
	
	If $t \in (\epsilon^k,\epsilon^{k+1})$, $t\cdot \epsilon^{-k} \in \mathbb { Z } [ \sqrt { d } ] ^ { \times , 1 }$. $t<\epsilon^{k+1} \implies 1=\epsilon^{0}<t\cdot \epsilon^{-k} < \epsilon$. Since $t \neq \pm \epsilon ^ { n } $, $t\cdot \epsilon^{-k}$ is a nontrivial element of $\mathbb { Z } [ \sqrt { d } ] ^ { \times , 1 }$. Contradiction, such $t$ must not exist.
	
	Similarly, if $-t \in (\epsilon^k,\epsilon^{k+1})$, such $t$ must not exist due to contradiction.
	
	Thus, every element of $\mathbb { Z } [ \sqrt { d } ] ^ { \times , 1 }$ is of the form $\pm \epsilon ^ { n }$ for some $n$ in $\mathbb { Z }$, where $\epsilon$ is the fundamental 1-unit.
\end{proof}		
\end{document}